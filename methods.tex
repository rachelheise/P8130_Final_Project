% Options for packages loaded elsewhere
\PassOptionsToPackage{unicode}{hyperref}
\PassOptionsToPackage{hyphens}{url}
%
\documentclass[
]{article}
\usepackage{lmodern}
\usepackage{amssymb,amsmath}
\usepackage{ifxetex,ifluatex}
\ifnum 0\ifxetex 1\fi\ifluatex 1\fi=0 % if pdftex
  \usepackage[T1]{fontenc}
  \usepackage[utf8]{inputenc}
  \usepackage{textcomp} % provide euro and other symbols
\else % if luatex or xetex
  \usepackage{unicode-math}
  \defaultfontfeatures{Scale=MatchLowercase}
  \defaultfontfeatures[\rmfamily]{Ligatures=TeX,Scale=1}
\fi
% Use upquote if available, for straight quotes in verbatim environments
\IfFileExists{upquote.sty}{\usepackage{upquote}}{}
\IfFileExists{microtype.sty}{% use microtype if available
  \usepackage[]{microtype}
  \UseMicrotypeSet[protrusion]{basicmath} % disable protrusion for tt fonts
}{}
\makeatletter
\@ifundefined{KOMAClassName}{% if non-KOMA class
  \IfFileExists{parskip.sty}{%
    \usepackage{parskip}
  }{% else
    \setlength{\parindent}{0pt}
    \setlength{\parskip}{6pt plus 2pt minus 1pt}}
}{% if KOMA class
  \KOMAoptions{parskip=half}}
\makeatother
\usepackage{xcolor}
\IfFileExists{xurl.sty}{\usepackage{xurl}}{} % add URL line breaks if available
\IfFileExists{bookmark.sty}{\usepackage{bookmark}}{\usepackage{hyperref}}
\hypersetup{
  pdftitle={Methods},
  hidelinks,
  pdfcreator={LaTeX via pandoc}}
\urlstyle{same} % disable monospaced font for URLs
\usepackage[margin=1in]{geometry}
\usepackage{color}
\usepackage{fancyvrb}
\newcommand{\VerbBar}{|}
\newcommand{\VERB}{\Verb[commandchars=\\\{\}]}
\DefineVerbatimEnvironment{Highlighting}{Verbatim}{commandchars=\\\{\}}
% Add ',fontsize=\small' for more characters per line
\usepackage{framed}
\definecolor{shadecolor}{RGB}{248,248,248}
\newenvironment{Shaded}{\begin{snugshade}}{\end{snugshade}}
\newcommand{\AlertTok}[1]{\textcolor[rgb]{0.94,0.16,0.16}{#1}}
\newcommand{\AnnotationTok}[1]{\textcolor[rgb]{0.56,0.35,0.01}{\textbf{\textit{#1}}}}
\newcommand{\AttributeTok}[1]{\textcolor[rgb]{0.77,0.63,0.00}{#1}}
\newcommand{\BaseNTok}[1]{\textcolor[rgb]{0.00,0.00,0.81}{#1}}
\newcommand{\BuiltInTok}[1]{#1}
\newcommand{\CharTok}[1]{\textcolor[rgb]{0.31,0.60,0.02}{#1}}
\newcommand{\CommentTok}[1]{\textcolor[rgb]{0.56,0.35,0.01}{\textit{#1}}}
\newcommand{\CommentVarTok}[1]{\textcolor[rgb]{0.56,0.35,0.01}{\textbf{\textit{#1}}}}
\newcommand{\ConstantTok}[1]{\textcolor[rgb]{0.00,0.00,0.00}{#1}}
\newcommand{\ControlFlowTok}[1]{\textcolor[rgb]{0.13,0.29,0.53}{\textbf{#1}}}
\newcommand{\DataTypeTok}[1]{\textcolor[rgb]{0.13,0.29,0.53}{#1}}
\newcommand{\DecValTok}[1]{\textcolor[rgb]{0.00,0.00,0.81}{#1}}
\newcommand{\DocumentationTok}[1]{\textcolor[rgb]{0.56,0.35,0.01}{\textbf{\textit{#1}}}}
\newcommand{\ErrorTok}[1]{\textcolor[rgb]{0.64,0.00,0.00}{\textbf{#1}}}
\newcommand{\ExtensionTok}[1]{#1}
\newcommand{\FloatTok}[1]{\textcolor[rgb]{0.00,0.00,0.81}{#1}}
\newcommand{\FunctionTok}[1]{\textcolor[rgb]{0.00,0.00,0.00}{#1}}
\newcommand{\ImportTok}[1]{#1}
\newcommand{\InformationTok}[1]{\textcolor[rgb]{0.56,0.35,0.01}{\textbf{\textit{#1}}}}
\newcommand{\KeywordTok}[1]{\textcolor[rgb]{0.13,0.29,0.53}{\textbf{#1}}}
\newcommand{\NormalTok}[1]{#1}
\newcommand{\OperatorTok}[1]{\textcolor[rgb]{0.81,0.36,0.00}{\textbf{#1}}}
\newcommand{\OtherTok}[1]{\textcolor[rgb]{0.56,0.35,0.01}{#1}}
\newcommand{\PreprocessorTok}[1]{\textcolor[rgb]{0.56,0.35,0.01}{\textit{#1}}}
\newcommand{\RegionMarkerTok}[1]{#1}
\newcommand{\SpecialCharTok}[1]{\textcolor[rgb]{0.00,0.00,0.00}{#1}}
\newcommand{\SpecialStringTok}[1]{\textcolor[rgb]{0.31,0.60,0.02}{#1}}
\newcommand{\StringTok}[1]{\textcolor[rgb]{0.31,0.60,0.02}{#1}}
\newcommand{\VariableTok}[1]{\textcolor[rgb]{0.00,0.00,0.00}{#1}}
\newcommand{\VerbatimStringTok}[1]{\textcolor[rgb]{0.31,0.60,0.02}{#1}}
\newcommand{\WarningTok}[1]{\textcolor[rgb]{0.56,0.35,0.01}{\textbf{\textit{#1}}}}
\usepackage{graphicx,grffile}
\makeatletter
\def\maxwidth{\ifdim\Gin@nat@width>\linewidth\linewidth\else\Gin@nat@width\fi}
\def\maxheight{\ifdim\Gin@nat@height>\textheight\textheight\else\Gin@nat@height\fi}
\makeatother
% Scale images if necessary, so that they will not overflow the page
% margins by default, and it is still possible to overwrite the defaults
% using explicit options in \includegraphics[width, height, ...]{}
\setkeys{Gin}{width=\maxwidth,height=\maxheight,keepaspectratio}
% Set default figure placement to htbp
\makeatletter
\def\fps@figure{htbp}
\makeatother
\setlength{\emergencystretch}{3em} % prevent overfull lines
\providecommand{\tightlist}{%
  \setlength{\itemsep}{0pt}\setlength{\parskip}{0pt}}
\setcounter{secnumdepth}{-\maxdimen} % remove section numbering

\title{Methods}
\author{}
\date{\vspace{-2.5em}}

\begin{document}
\maketitle

\hypertarget{data-cleaning}{%
\section{Data Cleaning}\label{data-cleaning}}

Read and clean hate crime data.

\begin{Shaded}
\begin{Highlighting}[]
\NormalTok{hate_df =}
\StringTok{  }\KeywordTok{read_csv}\NormalTok{(}\StringTok{"./data/HateCrimes.csv"}\NormalTok{) }\OperatorTok\StringTok{ }
\StringTok{  }\KeywordTok{mutate}\NormalTok{(}
    \DataTypeTok{state =} \KeywordTok{as.factor}\NormalTok{(state),}
    \DataTypeTok{unemployment =} \KeywordTok{as.factor}\NormalTok{(unemployment),}
    \DataTypeTok{urbanization =} \KeywordTok{as.factor}\NormalTok{(urbanization),}
    \DataTypeTok{hate_crimes_per_100k_splc =} \KeywordTok{as.numeric}\NormalTok{(hate_crimes_per_100k_splc)}
\NormalTok{  )}
\end{Highlighting}
\end{Shaded}

\hypertarget{descriptive-statistics}{%
\section{Descriptive Statistics}\label{descriptive-statistics}}

Create table of descriptive statistics.

\begin{Shaded}
\begin{Highlighting}[]
\CommentTok{# Table labels}
\NormalTok{my_labels =}
\StringTok{  }\KeywordTok{list}\NormalTok{(}
    \DataTypeTok{unemployment =} \StringTok{"Unemployment"}\NormalTok{,}
    \DataTypeTok{urbanization =} \StringTok{"Urbanization"}\NormalTok{,}
    \DataTypeTok{median_household_income =} \StringTok{"Median Household Income"}\NormalTok{,}
    \DataTypeTok{perc_population_with_high_school_degree =} \StringTok{"Percent with HS Degree"}\NormalTok{,}
    \DataTypeTok{perc_non_citizen =} \StringTok{"Percent Non-Citizen"}\NormalTok{,}
    \DataTypeTok{gini_index =} \StringTok{"Gini Index"}\NormalTok{,}
    \DataTypeTok{perc_non_white =} \StringTok{"Percent Non-White"}\NormalTok{,}
    \DataTypeTok{hate_crimes_per_100k_splc =} \StringTok{"Hate Crimes per 100k"}
\NormalTok{)}

\CommentTok{# Table controls}
\NormalTok{my_controls =}\StringTok{ }\KeywordTok{tableby.control}\NormalTok{(}
  \DataTypeTok{total =}\NormalTok{ F,}
  \DataTypeTok{test =}\NormalTok{ F,}
  \DataTypeTok{numeric.stats =} \KeywordTok{c}\NormalTok{(}\StringTok{"N"}\NormalTok{, }\StringTok{"meansd"}\NormalTok{, }\StringTok{"medianq1q3"}\NormalTok{, }\StringTok{"range"}\NormalTok{, }\StringTok{"Nmiss2"}\NormalTok{),}
  \DataTypeTok{cat.stats =} \KeywordTok{c}\NormalTok{(}\StringTok{"N"}\NormalTok{, }\StringTok{"countpct"}\NormalTok{),}
  \DataTypeTok{stats.labels =} \KeywordTok{list}\NormalTok{(}
    \DataTypeTok{meansd =} \StringTok{"Mean (SD)"}\NormalTok{,}
    \DataTypeTok{medianq1q3 =} \StringTok{"Median (Q1, Q3)"}\NormalTok{,}
    \DataTypeTok{range =} \StringTok{"Min - Max"}\NormalTok{,}
    \DataTypeTok{Nmiss2 =} \StringTok{"Missing"}\NormalTok{,}
    \DataTypeTok{countpct =} \StringTok{"N (%)"}\NormalTok{,}
    \DataTypeTok{N =} \StringTok{"N"}
\NormalTok{    )}
\NormalTok{  )}

\CommentTok{# Generate table}
\NormalTok{descriptive_tab =}
\StringTok{  }\KeywordTok{tableby}\NormalTok{( }\OperatorTok{~}\StringTok{ }\NormalTok{unemployment }\OperatorTok{+}\StringTok{ }
\StringTok{           }\NormalTok{urbanization }\OperatorTok{+}
\StringTok{           }\NormalTok{median_household_income }\OperatorTok{+}
\StringTok{           }\NormalTok{perc_population_with_high_school_degree }\OperatorTok{+}
\StringTok{           }\NormalTok{perc_non_citizen }\OperatorTok{+}
\StringTok{           }\NormalTok{gini_index }\OperatorTok{+}
\StringTok{           }\NormalTok{perc_non_white }\OperatorTok{+}
\StringTok{           }\NormalTok{hate_crimes_per_100k_splc,}
           \DataTypeTok{data =}\NormalTok{ hate_df,}
           \DataTypeTok{control =}\NormalTok{ my_controls)}

\KeywordTok{summary}\NormalTok{(}
\NormalTok{  descriptive_tab,}
  \DataTypeTok{title =} \StringTok{"Table 1: Descriptive Statistics: Hate Crimes Data"}\NormalTok{,}
  \DataTypeTok{labelTranslations =}\NormalTok{ my_labels,}
  \DataTypeTok{text =}\NormalTok{ T)}
\end{Highlighting}
\end{Shaded}

\begin{verbatim}
## 
## Table: Table 1: Descriptive Statistics: Hate Crimes Data
## 
## |                        |          Overall (N=51)          |
## |:-----------------------|:--------------------------------:|
## |Unemployment            |                                  |
## |-  N                    |                51                |
## |-  high                 |            24 (47.1%)            |
## |-  low                  |            27 (52.9%)            |
## |Urbanization            |                                  |
## |-  N                    |                51                |
## |-  high                 |            24 (47.1%)            |
## |-  low                  |            27 (52.9%)            |
## |Median Household Income |                                  |
## |-  N                    |                51                |
## |-  Mean (SD)            |       55223.608 (9208.478)       |
## |-  Median (Q1, Q3)      | 54916.000 (48657.000, 60719.000) |
## |-  Min - Max            |      35521.000 - 76165.000       |
## |-  Missing              |                0                 |
## |Percent with HS Degree  |                                  |
## |-  N                    |                51                |
## |-  Mean (SD)            |          0.869 (0.034)           |
## |-  Median (Q1, Q3)      |       0.874 (0.841, 0.898)       |
## |-  Min - Max            |          0.799 - 0.918           |
## |-  Missing              |                0                 |
## |Percent Non-Citizen     |                                  |
## |-  N                    |                48                |
## |-  Mean (SD)            |          0.055 (0.031)           |
## |-  Median (Q1, Q3)      |       0.045 (0.030, 0.080)       |
## |-  Min - Max            |          0.010 - 0.130           |
## |-  Missing              |                3                 |
## |Gini Index              |                                  |
## |-  N                    |                51                |
## |-  Mean (SD)            |          0.454 (0.021)           |
## |-  Median (Q1, Q3)      |       0.454 (0.440, 0.467)       |
## |-  Min - Max            |          0.419 - 0.532           |
## |-  Missing              |                0                 |
## |Percent Non-White       |                                  |
## |-  N                    |                51                |
## |-  Mean (SD)            |          0.316 (0.165)           |
## |-  Median (Q1, Q3)      |       0.280 (0.195, 0.420)       |
## |-  Min - Max            |          0.060 - 0.810           |
## |-  Missing              |                0                 |
## |Hate Crimes per 100k    |                                  |
## |-  N                    |                47                |
## |-  Mean (SD)            |          0.304 (0.253)           |
## |-  Median (Q1, Q3)      |       0.226 (0.143, 0.357)       |
## |-  Min - Max            |          0.067 - 1.522           |
## |-  Missing              |                4                 |
\end{verbatim}

\hypertarget{remove-missing-data}{%
\section{Remove Missing Data}\label{remove-missing-data}}

\begin{Shaded}
\begin{Highlighting}[]
\NormalTok{hate_nona_df =}\StringTok{ }\CommentTok{# Removing rows with missing values from the dataset}
\StringTok{  }\NormalTok{hate_df }\OperatorTok\StringTok{ }
\StringTok{  }\KeywordTok{drop_na}\NormalTok{()}
\end{Highlighting}
\end{Shaded}

\hypertarget{distribution-of-outcome-data}{%
\section{Distribution of Outcome
Data}\label{distribution-of-outcome-data}}

Plot a histogram of raw outcome data (hate crimes per 100k) to assess
distribution shape.

\begin{Shaded}
\begin{Highlighting}[]
\NormalTok{hate_df }\OperatorTok\StringTok{ }
\StringTok{  }\KeywordTok{ggplot}\NormalTok{(}\KeywordTok{aes}\NormalTok{(}\DataTypeTok{x =}\NormalTok{ hate_crimes_per_100k_splc, }\DataTypeTok{y =}\NormalTok{ ..density..)) }\OperatorTok{+}
\StringTok{  }\KeywordTok{geom_histogram}\NormalTok{(}\DataTypeTok{bins =} \DecValTok{11}\NormalTok{) }\OperatorTok{+}
\StringTok{  }\KeywordTok{geom_density}\NormalTok{(}\DataTypeTok{alpha =} \FloatTok{0.2}\NormalTok{, }\DataTypeTok{color =} \StringTok{"blue"}\NormalTok{) }\OperatorTok{+}
\StringTok{  }\KeywordTok{labs}\NormalTok{(}
    \DataTypeTok{x =} \StringTok{"Hate Crimes per 100k Population"}\NormalTok{,}
    \DataTypeTok{y =} \StringTok{"Density"}\NormalTok{,}
    \DataTypeTok{title =} \StringTok{"Histogram of Hate Crimes Data"}
\NormalTok{  )}
\end{Highlighting}
\end{Shaded}

\includegraphics{methods_files/figure-latex/unnamed-chunk-4-1.pdf}

These data look skewed, so we use the Box-Cox method to determine the
best transformation for the data. First, fit a linear model with all
main effects.

\begin{Shaded}
\begin{Highlighting}[]
\NormalTok{full_lm =}\StringTok{ }\KeywordTok{lm}\NormalTok{(}
\NormalTok{  hate_crimes_per_100k_splc}
  \OperatorTok{~}\StringTok{ }\NormalTok{unemployment }\OperatorTok{+}\StringTok{ }
\StringTok{    }\NormalTok{urbanization }\OperatorTok{+}
\StringTok{    }\NormalTok{median_household_income }\OperatorTok{+}
\StringTok{    }\NormalTok{perc_population_with_high_school_degree }\OperatorTok{+}
\StringTok{    }\NormalTok{perc_non_citizen }\OperatorTok{+}
\StringTok{    }\NormalTok{gini_index }\OperatorTok{+}
\StringTok{    }\NormalTok{perc_non_white,}
  \DataTypeTok{data =}\NormalTok{ hate_nona_df}
\NormalTok{)}
\end{Highlighting}
\end{Shaded}

Run boxcox on this model.

\begin{Shaded}
\begin{Highlighting}[]
\NormalTok{MASS}\OperatorTok{::}\KeywordTok{boxcox}\NormalTok{(full_lm)}
\end{Highlighting}
\end{Shaded}

\includegraphics{methods_files/figure-latex/unnamed-chunk-6-1.pdf}

The optimal value of \(Y^a\) is near 0, indicating that a natural log
transformation of the outcome for all practical intents and purposes is
best. We proceed with the log transformation.

Create a histogram of log-transformed outcome data (hate crimes per
100k).

\begin{Shaded}
\begin{Highlighting}[]
\NormalTok{hate_df }\OperatorTok\StringTok{ }
\StringTok{  }\KeywordTok{ggplot}\NormalTok{(}\KeywordTok{aes}\NormalTok{(}\DataTypeTok{x =} \KeywordTok{log}\NormalTok{(hate_crimes_per_100k_splc), }\DataTypeTok{y =}\NormalTok{ ..density..)) }\OperatorTok{+}
\StringTok{  }\KeywordTok{geom_histogram}\NormalTok{(}\DataTypeTok{bins =} \DecValTok{11}\NormalTok{) }\OperatorTok{+}
\StringTok{  }\KeywordTok{geom_density}\NormalTok{(}\DataTypeTok{alpha =} \FloatTok{0.2}\NormalTok{, }\DataTypeTok{color =} \StringTok{"blue"}\NormalTok{) }\OperatorTok{+}
\StringTok{  }\KeywordTok{labs}\NormalTok{(}
    \DataTypeTok{x =} \StringTok{"Hate Crimes per 100k (Log Scale)"}\NormalTok{,}
    \DataTypeTok{y =} \StringTok{"Density"}\NormalTok{,}
    \DataTypeTok{title =} \StringTok{"Histogram of Hate Crimes Per 100k on Log Scale"}
\NormalTok{  )}
\end{Highlighting}
\end{Shaded}

\includegraphics{methods_files/figure-latex/unnamed-chunk-7-1.pdf}

This plot looks less skewed.

\hypertarget{examining-potential-multicollinearity}{%
\section{Examining Potential
Multicollinearity}\label{examining-potential-multicollinearity}}

Examine correlations between predictors.

\begin{Shaded}
\begin{Highlighting}[]
\NormalTok{hate_df }\OperatorTok\StringTok{ }
\StringTok{  }\KeywordTok{select}\NormalTok{(}
\NormalTok{    hate_crimes_per_100k_splc,}
\NormalTok{    median_household_income,}
\NormalTok{    perc_population_with_high_school_degree,}
\NormalTok{    perc_non_citizen,}
\NormalTok{    gini_index,}
\NormalTok{    perc_non_white}
\NormalTok{    ) }\OperatorTok
\StringTok{  }\KeywordTok{cor}\NormalTok{(}\DataTypeTok{use =} \StringTok{"complete.obs"}\NormalTok{) }\OperatorTok\StringTok{ }\CommentTok{# Ignoring NA values}
\StringTok{  }\KeywordTok{round}\NormalTok{(., }\DecValTok{2}\NormalTok{)}
\end{Highlighting}
\end{Shaded}

\begin{verbatim}
##                                         hate_crimes_per_100k_splc
## hate_crimes_per_100k_splc                                    1.00
## median_household_income                                      0.34
## perc_population_with_high_school_degree                      0.26
## perc_non_citizen                                             0.24
## gini_index                                                   0.38
## perc_non_white                                               0.11
##                                         median_household_income
## hate_crimes_per_100k_splc                                  0.34
## median_household_income                                    1.00
## perc_population_with_high_school_degree                    0.65
## perc_non_citizen                                           0.30
## gini_index                                                -0.13
## perc_non_white                                             0.04
##                                         perc_population_with_high_school_degree
## hate_crimes_per_100k_splc                                                  0.26
## median_household_income                                                    0.65
## perc_population_with_high_school_degree                                    1.00
## perc_non_citizen                                                          -0.26
## gini_index                                                                -0.54
## perc_non_white                                                            -0.50
##                                         perc_non_citizen gini_index
## hate_crimes_per_100k_splc                           0.24       0.38
## median_household_income                             0.30      -0.13
## perc_population_with_high_school_degree            -0.26      -0.54
## perc_non_citizen                                    1.00       0.48
## gini_index                                          0.48       1.00
## perc_non_white                                      0.75       0.55
##                                         perc_non_white
## hate_crimes_per_100k_splc                         0.11
## median_household_income                           0.04
## perc_population_with_high_school_degree          -0.50
## perc_non_citizen                                  0.75
## gini_index                                        0.55
## perc_non_white                                    1.00
\end{verbatim}

Based on this output, the following pairs of variables have a
correlation of 60\% or higher:

\begin{itemize}
\item
  Percentage non-citizens \& percentage non-white (0.75)
\item
  Median household income \& percentage of population with a high school
  degree (0.65)
\end{itemize}

Use a pairs plot to visually assess potential multicollinearity.

\begin{Shaded}
\begin{Highlighting}[]
\NormalTok{hate_df }\OperatorTok\StringTok{ }
\StringTok{  }\KeywordTok{select}\NormalTok{(}
\NormalTok{    hate_crimes_per_100k_splc,}
\NormalTok{    median_household_income,}
\NormalTok{    perc_population_with_high_school_degree,}
\NormalTok{    perc_non_citizen,}
\NormalTok{    gini_index,}
\NormalTok{    perc_non_white}
\NormalTok{    ) }\OperatorTok\StringTok{ }
\StringTok{  }\KeywordTok{pairs}\NormalTok{()}
\end{Highlighting}
\end{Shaded}

\includegraphics{methods_files/figure-latex/unnamed-chunk-9-1.pdf}

\hypertarget{selecting-a-model-using-the-stepwise-approach}{%
\section{Selecting a Model Using the Stepwise
Approach}\label{selecting-a-model-using-the-stepwise-approach}}

First, consider all main effects in the model, using a log
transformation of the outcome.

\begin{Shaded}
\begin{Highlighting}[]
\NormalTok{full_log_lm =}\StringTok{ }\KeywordTok{lm}\NormalTok{(}
  \KeywordTok{log}\NormalTok{(hate_crimes_per_100k_splc)}
  \OperatorTok{~}\StringTok{ }\NormalTok{unemployment }\OperatorTok{+}\StringTok{ }
\StringTok{    }\NormalTok{urbanization }\OperatorTok{+}
\StringTok{    }\NormalTok{median_household_income }\OperatorTok{+}
\StringTok{    }\NormalTok{perc_population_with_high_school_degree }\OperatorTok{+}
\StringTok{    }\NormalTok{perc_non_citizen }\OperatorTok{+}
\StringTok{    }\NormalTok{gini_index }\OperatorTok{+}
\StringTok{    }\NormalTok{perc_non_white,}
  \DataTypeTok{data =}\NormalTok{ hate_nona_df}
\NormalTok{)}

\KeywordTok{summary}\NormalTok{(full_log_lm)}
\end{Highlighting}
\end{Shaded}

\begin{verbatim}
## 
## Call:
## lm(formula = log(hate_crimes_per_100k_splc) ~ unemployment + 
##     urbanization + median_household_income + perc_population_with_high_school_degree + 
##     perc_non_citizen + gini_index + perc_non_white, data = hate_nona_df)
## 
## Residuals:
##      Min       1Q   Median       3Q      Max 
## -1.28845 -0.41144  0.01898  0.31334  1.13022 
## 
## Coefficients:
##                                           Estimate Std. Error t value Pr(>|t|)
## (Intercept)                             -1.857e+01  5.553e+00  -3.344  0.00190
## unemploymentlow                          2.179e-01  2.088e-01   1.043  0.30353
## urbanizationlow                         -9.885e-02  2.467e-01  -0.401  0.69092
## median_household_income                 -4.732e-06  1.735e-05  -0.273  0.78658
## perc_population_with_high_school_degree  1.121e+01  5.341e+00   2.098  0.04275
## perc_non_citizen                         1.168e+00  5.464e+00   0.214  0.83189
## gini_index                               1.670e+01  5.744e+00   2.908  0.00611
## perc_non_white                          -1.232e-01  1.069e+00  -0.115  0.90887
##                                           
## (Intercept)                             **
## unemploymentlow                           
## urbanizationlow                           
## median_household_income                   
## perc_population_with_high_school_degree * 
## perc_non_citizen                          
## gini_index                              **
## perc_non_white                            
## ---
## Signif. codes:  0 '***' 0.001 '**' 0.01 '*' 0.05 '.' 0.1 ' ' 1
## 
## Residual standard error: 0.5862 on 37 degrees of freedom
## Multiple R-squared:  0.3146, Adjusted R-squared:  0.1849 
## F-statistic: 2.426 on 7 and 37 DF,  p-value: 0.03768
\end{verbatim}

Use the Stepwise approach.

\begin{Shaded}
\begin{Highlighting}[]
\KeywordTok{step}\NormalTok{(full_log_lm, }\DataTypeTok{direction =} \StringTok{"both"}\NormalTok{)}
\end{Highlighting}
\end{Shaded}

This procedure retains the following two predictors:

\begin{itemize}
\item
  Precent population with high school degree
\item
  Gini index
\end{itemize}

Fit a linear regression model based on the results of the stepwise
procedure.

\begin{Shaded}
\begin{Highlighting}[]
\NormalTok{stepwise_log_lm =}\StringTok{ }\KeywordTok{lm}\NormalTok{(}
  \KeywordTok{log}\NormalTok{(hate_crimes_per_100k_splc)}
  \OperatorTok{~}\StringTok{ }\NormalTok{perc_population_with_high_school_degree }\OperatorTok{+}
\StringTok{    }\NormalTok{gini_index,}
  \DataTypeTok{data =}\NormalTok{ hate_nona_df)}
\end{Highlighting}
\end{Shaded}

Check model assumptions using this model.

\begin{Shaded}
\begin{Highlighting}[]
\KeywordTok{par}\NormalTok{(}\DataTypeTok{mfrow =} \KeywordTok{c}\NormalTok{(}\DecValTok{2}\NormalTok{, }\DecValTok{2}\NormalTok{))}
\KeywordTok{plot}\NormalTok{(stepwise_log_lm)}
\end{Highlighting}
\end{Shaded}

\includegraphics{methods_files/figure-latex/unnamed-chunk-12-1.pdf}

\hypertarget{influential-points}{%
\section{Influential Points}\label{influential-points}}

Determine if DC is an influential point quantitatively using the Cook's
value. Note that on the previous assumptions plot, DC is point 9 and
looks like it has a high Cook's Distance.

Check for influential points and create a dataframe without the
Washington, DC point to see what impact that has on the model.

\begin{Shaded}
\begin{Highlighting}[]
\KeywordTok{influence.measures}\NormalTok{(stepwise_log_lm)}
\end{Highlighting}
\end{Shaded}

\begin{verbatim}
## Influence measures of
##   lm(formula = log(hate_crimes_per_100k_splc) ~ perc_population_with_high_school_degree +      gini_index, data = hate_nona_df) :
## 
##       dfb.1_ dfb.p___  dfb.gn_n    dffit cov.r   cook.d    hat inf
## 1  -0.081618  0.13038 -4.05e-03 -0.19261 1.109 1.25e-02 0.0666    
## 2  -0.013893 -0.10481  1.49e-01 -0.30406 1.109 3.09e-02 0.0922    
## 3   0.054057 -0.05763 -3.27e-02  0.08493 1.108 2.45e-03 0.0412    
## 4  -0.304896  0.34577  1.64e-01 -0.41699 0.987 5.63e-02 0.0720    
## 5   0.191496 -0.26261 -4.75e-02  0.32205 1.122 3.47e-02 0.1027    
## 6  -0.039118  0.04304  2.54e-02  0.06064 1.119 1.25e-03 0.0450    
## 7   0.251790 -0.19747 -2.54e-01 -0.29040 1.186 2.84e-02 0.1302    
## 8   0.068771 -0.02700 -9.62e-02  0.16236 1.060 8.85e-03 0.0360    
## 9  -1.027893  0.67191  1.19e+00  1.22378 1.778 4.91e-01 0.4655   *
## 10  0.046163 -0.00709 -8.54e-02 -0.14609 1.078 7.20e-03 0.0398    
## 11 -0.053388  0.10169 -2.60e-02 -0.21260 1.032 1.50e-02 0.0390    
## 12 -0.075257  0.00825  1.31e-01 -0.20352 1.067 1.39e-02 0.0491    
## 13  0.031597 -0.01738 -4.38e-02 -0.10108 1.077 3.46e-03 0.0275    
## 14  0.063828 -0.04013 -7.24e-02  0.10745 1.099 3.92e-03 0.0407    
## 15 -0.024348  0.13848 -1.15e-01  0.30940 1.082 3.18e-02 0.0813    
## 16  0.133702 -0.22611  3.91e-03 -0.39474 0.874 4.90e-02 0.0419    
## 17  0.235394 -0.30556 -7.86e-02  0.38074 1.031 4.75e-02 0.0780    
## 18 -0.093402  0.17570 -3.85e-02 -0.28997 1.056 2.79e-02 0.0658    
## 19 -0.013703  0.04710 -2.68e-02  0.12712 1.079 5.46e-03 0.0355    
## 20 -0.182960  0.15692  1.69e-01  0.21549 1.139 1.57e-02 0.0886    
## 21 -0.019768  0.03790 -2.98e-03  0.13067 1.052 5.74e-03 0.0255    
## 22 -0.154266  0.24423  1.03e-02  0.34035 1.041 3.82e-02 0.0722    
## 23 -0.000508 -0.00222  8.76e-04 -0.07363 1.081 1.84e-03 0.0223    
## 24 -0.056507  0.14587 -6.12e-02  0.27564 1.054 2.52e-02 0.0618    
## 25 -0.014601 -0.04569  7.99e-02 -0.17059 1.098 9.82e-03 0.0554    
## 26 -0.031721  0.03033  2.39e-02 -0.03988 1.142 5.43e-04 0.0600    
## 27  0.002872 -0.10873  1.21e-01 -0.27889 1.104 2.60e-02 0.0840    
## 28  0.248803 -0.20839 -2.46e-01 -0.47387 0.701 6.58e-02 0.0320   *
## 29  0.127658 -0.16623 -4.07e-02  0.22682 1.069 1.72e-02 0.0558    
## 30  0.049492 -0.02110 -7.20e-02 -0.08242 1.229 2.32e-03 0.1282   *
## 31  0.015806 -0.02440  4.52e-04  0.04951 1.107 8.36e-04 0.0341    
## 32  0.008525 -0.01768  2.00e-03 -0.07960 1.081 2.15e-03 0.0241    
## 33 -0.054224  0.05286  3.67e-02 -0.13131 1.055 5.80e-03 0.0267    
## 34 -0.140534  0.20997  2.97e-02  0.38273 0.841 4.55e-02 0.0351    
## 35  0.011274 -0.01075 -9.53e-03 -0.02090 1.110 1.49e-04 0.0322    
## 36 -0.029075  0.09228 -6.52e-02 -0.29814 0.917 2.85e-02 0.0319    
## 37  0.026918 -0.03342 -1.04e-02  0.04908 1.120 8.21e-04 0.0442    
## 38  0.003366 -0.00523  2.32e-05  0.00839 1.131 2.40e-05 0.0492    
## 39  0.214374 -0.27630 -7.60e-02  0.31860 1.170 3.41e-02 0.1268    
## 40 -0.065943 -0.02807  1.58e-01 -0.23735 1.140 1.90e-02 0.0936    
## 41  0.003208 -0.00467 -7.03e-04 -0.00639 1.147 1.39e-05 0.0631    
## 42 -0.011898  0.00781  1.74e-02  0.09922 1.067 3.33e-03 0.0229    
## 43 -0.070951  0.17415 -6.38e-02  0.36609 0.911 4.27e-02 0.0432    
## 44  0.372189 -0.38111 -2.51e-01  0.45593 0.978 6.69e-02 0.0776    
## 45 -0.000251 -0.00039  9.22e-04 -0.00177 1.143 1.07e-06 0.0595
\end{verbatim}

\begin{Shaded}
\begin{Highlighting}[]
\NormalTok{stu_res <-}\StringTok{ }\KeywordTok{rstandard}\NormalTok{(stepwise_log_lm)}
\NormalTok{outliers_y <-}\StringTok{ }\NormalTok{stu_res[}\KeywordTok{abs}\NormalTok{(stu_res) }\OperatorTok{>}\StringTok{ }\FloatTok{2.5}\NormalTok{]}
\NormalTok{outliers_y}
\end{Highlighting}
\end{Shaded}

\begin{verbatim}
## named numeric(0)
\end{verbatim}

\begin{Shaded}
\begin{Highlighting}[]
\KeywordTok{summary}\NormalTok{(stepwise_log_lm)}
\end{Highlighting}
\end{Shaded}

\begin{verbatim}
## 
## Call:
## lm(formula = log(hate_crimes_per_100k_splc) ~ perc_population_with_high_school_degree + 
##     gini_index, data = hate_nona_df)
## 
## Residuals:
##      Min       1Q   Median       3Q      Max 
## -1.34787 -0.39659 -0.00387  0.44892  1.06723 
## 
## Coefficients:
##                                         Estimate Std. Error t value Pr(>|t|)
## (Intercept)                              -18.947      4.254  -4.454 6.14e-05
## perc_population_with_high_school_degree   11.554      3.069   3.765 0.000512
## gini_index                                16.486      4.795   3.438 0.001334
##                                            
## (Intercept)                             ***
## perc_population_with_high_school_degree ***
## gini_index                              ** 
## ---
## Signif. codes:  0 '***' 0.001 '**' 0.01 '*' 0.05 '.' 0.1 ' ' 1
## 
## Residual standard error: 0.5608 on 42 degrees of freedom
## Multiple R-squared:  0.288,  Adjusted R-squared:  0.2541 
## F-statistic: 8.496 on 2 and 42 DF,  p-value: 0.0007974
\end{verbatim}

\begin{Shaded}
\begin{Highlighting}[]
\NormalTok{full_nodc =}\StringTok{ }\KeywordTok{lm}\NormalTok{(}\KeywordTok{log}\NormalTok{(hate_crimes_per_100k_splc) }\OperatorTok{~}\StringTok{ }\NormalTok{perc_population_with_high_school_degree }\OperatorTok{+}\StringTok{ }\NormalTok{gini_index, }\DataTypeTok{data =}\NormalTok{ hate_nona_df[}\OperatorTok{-}\DecValTok{9}\NormalTok{,])}
\KeywordTok{summary}\NormalTok{(full_nodc)}
\end{Highlighting}
\end{Shaded}

\begin{verbatim}
## 
## Call:
## lm(formula = log(hate_crimes_per_100k_splc) ~ perc_population_with_high_school_degree + 
##     gini_index, data = hate_nona_df[-9, ])
## 
## Residuals:
##      Min       1Q   Median       3Q      Max 
## -1.26341 -0.43818  0.03534  0.42669  1.10132 
## 
## Coefficients:
##                                         Estimate Std. Error t value Pr(>|t|)   
## (Intercept)                              -14.611      5.359  -2.726  0.00938 **
## perc_population_with_high_school_degree    9.509      3.419   2.781  0.00814 **
## gini_index                                10.811      6.429   1.682  0.10025   
## ---
## Signif. codes:  0 '***' 0.001 '**' 0.01 '*' 0.05 '.' 0.1 ' ' 1
## 
## Residual standard error: 0.556 on 41 degrees of freedom
## Multiple R-squared:  0.1595, Adjusted R-squared:  0.1185 
## F-statistic: 3.889 on 2 and 41 DF,  p-value: 0.02841
\end{verbatim}

\begin{Shaded}
\begin{Highlighting}[]
\CommentTok{# Re-fit model without Gini index variable}
\NormalTok{hsdeg_lm_nodc =}\StringTok{ }\KeywordTok{lm}\NormalTok{(}\KeywordTok{log}\NormalTok{(hate_crimes_per_100k_splc) }\OperatorTok{~}\StringTok{ }\NormalTok{perc_population_with_high_school_degree, }\DataTypeTok{data =}\NormalTok{ hate_nona_df[}\OperatorTok{-}\DecValTok{9}\NormalTok{,])}
\KeywordTok{summary}\NormalTok{(hsdeg_lm_nodc)}
\end{Highlighting}
\end{Shaded}

\begin{verbatim}
## 
## Call:
## lm(formula = log(hate_crimes_per_100k_splc) ~ perc_population_with_high_school_degree, 
##     data = hate_nona_df[-9, ])
## 
## Residuals:
##      Min       1Q   Median       3Q      Max 
## -1.12633 -0.48978  0.08266  0.44786  1.14080 
## 
## Coefficients:
##                                         Estimate Std. Error t value Pr(>|t|)   
## (Intercept)                               -6.413      2.274  -2.820   0.0073 **
## perc_population_with_high_school_degree    5.712      2.623   2.178   0.0351 * 
## ---
## Signif. codes:  0 '***' 0.001 '**' 0.01 '*' 0.05 '.' 0.1 ' ' 1
## 
## Residual standard error: 0.568 on 42 degrees of freedom
## Multiple R-squared:  0.1015, Adjusted R-squared:  0.08009 
## F-statistic: 4.744 on 1 and 42 DF,  p-value: 0.03507
\end{verbatim}

Using Cook's Distance and studentized residuals, DC could be an
influential point. Cook's Distance is 0.491, close to the threshold of
0.5. The studentized residual is not greater than 2.5 for any variable.
However, DFFIT is greater than 1 for DC which could be cause for
concern. Comparing the regression analysis shows that this point is
influential, so we need to consider deleting it.

Check model assumptions after removing DC as a point.

\begin{Shaded}
\begin{Highlighting}[]
\KeywordTok{par}\NormalTok{(}\DataTypeTok{mfrow =} \KeywordTok{c}\NormalTok{(}\DecValTok{2}\NormalTok{, }\DecValTok{2}\NormalTok{))}
\KeywordTok{plot}\NormalTok{(full_nodc)}
\end{Highlighting}
\end{Shaded}

\includegraphics{methods_files/figure-latex/unnamed-chunk-14-1.pdf} See
if Gini index is significant by itself, with a log transformed outcome
and DC point removed.

\begin{Shaded}
\begin{Highlighting}[]
\NormalTok{gini_nodc_lm =}\StringTok{ }\KeywordTok{lm}\NormalTok{(}\KeywordTok{log}\NormalTok{(hate_crimes_per_100k_splc) }\OperatorTok{~}\StringTok{ }\NormalTok{gini_index, }\DataTypeTok{data =}\NormalTok{ hate_nona_df[}\OperatorTok{-}\DecValTok{9}\NormalTok{,])}
\KeywordTok{summary}\NormalTok{(gini_nodc_lm)}
\end{Highlighting}
\end{Shaded}

\begin{verbatim}
## 
## Call:
## lm(formula = log(hate_crimes_per_100k_splc) ~ gini_index, data = hate_nona_df[-9, 
##     ])
## 
## Residuals:
##      Min       1Q   Median       3Q      Max 
## -1.20509 -0.49941 -0.03609  0.41639  1.27579 
## 
## Coefficients:
##             Estimate Std. Error t value Pr(>|t|)
## (Intercept)  -1.0103     2.3622  -0.428    0.671
## gini_index   -0.9987     5.1997  -0.192    0.849
## 
## Residual standard error: 0.5989 on 42 degrees of freedom
## Multiple R-squared:  0.0008775,  Adjusted R-squared:  -0.02291 
## F-statistic: 0.03689 on 1 and 42 DF,  p-value: 0.8486
\end{verbatim}

Gini index is not significant in this model.

\hypertarget{interactions}{%
\section{Interactions}\label{interactions}}

Check for interactions.

Fit a linear model with all two-way interactions and investigate which
interactions could be significant.

\begin{Shaded}
\begin{Highlighting}[]
\NormalTok{full_int_lm =}\StringTok{ }\KeywordTok{lm}\NormalTok{(}\KeywordTok{log}\NormalTok{(hate_crimes_per_100k_splc)}\OperatorTok{~}\NormalTok{(.}\OperatorTok{-}\NormalTok{state)}\OperatorTok{^}\DecValTok{2}\NormalTok{, }\DataTypeTok{data =}\NormalTok{ hate_nona_df)}
\KeywordTok{summary}\NormalTok{(full_int_lm)}
\end{Highlighting}
\end{Shaded}

\begin{Shaded}
\begin{Highlighting}[]
\NormalTok{hate_nona_df }\OperatorTok\StringTok{ }
\StringTok{  }\KeywordTok{ggplot}\NormalTok{(}\KeywordTok{aes}\NormalTok{(}\DataTypeTok{x =}\NormalTok{ median_household_income, }\DataTypeTok{y =} \KeywordTok{log}\NormalTok{(hate_crimes_per_100k_splc), }\DataTypeTok{color =}\NormalTok{ unemployment)) }\OperatorTok{+}
\StringTok{  }\KeywordTok{geom_point}\NormalTok{() }\OperatorTok{+}
\StringTok{  }\KeywordTok{geom_smooth}\NormalTok{(}\DataTypeTok{method =} \StringTok{"lm"}\NormalTok{, }\DataTypeTok{se =} \OtherTok{FALSE}\NormalTok{) }\OperatorTok{+}
\StringTok{  }\KeywordTok{labs}\NormalTok{(}\DataTypeTok{x =} \StringTok{"Median Household Income"}\NormalTok{,}
       \DataTypeTok{y =} \StringTok{"log(Hate Crimes Per 100k splc)"}\NormalTok{,}
       \DataTypeTok{title =} \StringTok{"Interaction Plot for Unemployment and Median Household Income"}\NormalTok{)}
\end{Highlighting}
\end{Shaded}

\begin{verbatim}
## `geom_smooth()` using formula 'y ~ x'
\end{verbatim}

\includegraphics{methods_files/figure-latex/unnamed-chunk-17-1.pdf}

\begin{Shaded}
\begin{Highlighting}[]
\NormalTok{hate_nona_df }\OperatorTok\StringTok{ }
\StringTok{  }\KeywordTok{ggplot}\NormalTok{(}\KeywordTok{aes}\NormalTok{(}\DataTypeTok{x =}\NormalTok{ median_household_income, }\DataTypeTok{y =} \KeywordTok{log}\NormalTok{(hate_crimes_per_100k_splc), }\DataTypeTok{color =}\NormalTok{ urbanization)) }\OperatorTok{+}
\StringTok{  }\KeywordTok{geom_point}\NormalTok{() }\OperatorTok{+}
\StringTok{  }\KeywordTok{geom_smooth}\NormalTok{(}\DataTypeTok{method =} \StringTok{"lm"}\NormalTok{, }\DataTypeTok{se =} \OtherTok{FALSE}\NormalTok{) }\OperatorTok{+}
\StringTok{  }\KeywordTok{labs}\NormalTok{(}\DataTypeTok{x =} \StringTok{"Median Household Income"}\NormalTok{,}
       \DataTypeTok{y =} \StringTok{"log(Hate Crimes Per 100k splc)"}\NormalTok{,}
       \DataTypeTok{title =} \StringTok{"Interaction Plot for Urbanization and Median Household Income"}\NormalTok{)}
\end{Highlighting}
\end{Shaded}

\begin{verbatim}
## `geom_smooth()` using formula 'y ~ x'
\end{verbatim}

\includegraphics{methods_files/figure-latex/unnamed-chunk-17-2.pdf}

There were three significant interactions, two of which occur between a
categorical and continuous variable, between median household income and
unemployment, and median household income and urbanization. Looking at
the interaction plots between these variables we do see that the lines
for the two levels of the categorical variables cross, indicating
interaction.

Perform a stratified analysis on these

\begin{Shaded}
\begin{Highlighting}[]
\CommentTok{# Stratified analysis for unemployment}

\NormalTok{unemployment_low =}\StringTok{ }\NormalTok{hate_nona_df }\OperatorTok\StringTok{ }
\StringTok{  }\KeywordTok{filter}\NormalTok{(unemployment }\OperatorTok{==}\StringTok{ "low"}\NormalTok{) }\OperatorTok\StringTok{ }
\StringTok{  }\KeywordTok{select}\NormalTok{(}\OperatorTok{-}\NormalTok{unemployment)}

\NormalTok{unemployment_high =}\StringTok{ }\NormalTok{hate_nona_df }\OperatorTok\StringTok{ }
\StringTok{  }\KeywordTok{filter}\NormalTok{(unemployment }\OperatorTok{==}\StringTok{ "high"}\NormalTok{) }\OperatorTok\StringTok{ }
\StringTok{  }\KeywordTok{select}\NormalTok{(}\OperatorTok{-}\NormalTok{unemployment)}

\NormalTok{unemployment_low_lm =}\StringTok{ }\KeywordTok{lm}\NormalTok{(}\KeywordTok{log}\NormalTok{(hate_crimes_per_100k_splc)}\OperatorTok{~}\NormalTok{.}\OperatorTok{-}\NormalTok{state, }\DataTypeTok{data =}\NormalTok{ unemployment_low)}
\KeywordTok{summary}\NormalTok{(unemployment_low_lm)}
\end{Highlighting}
\end{Shaded}

\begin{verbatim}
## 
## Call:
## lm(formula = log(hate_crimes_per_100k_splc) ~ . - state, data = unemployment_low)
## 
## Residuals:
##      Min       1Q   Median       3Q      Max 
## -0.81949 -0.33627  0.04041  0.26165  0.78586 
## 
## Coefficients:
##                                           Estimate Std. Error t value Pr(>|t|)
## (Intercept)                             -1.499e+01  8.468e+00  -1.770    0.097
## urbanizationlow                         -3.689e-01  3.455e-01  -1.068    0.303
## median_household_income                  1.328e-05  2.279e-05   0.582    0.569
## perc_population_with_high_school_degree  5.270e+00  7.077e+00   0.745    0.468
## perc_non_citizen                        -5.389e+00  1.040e+01  -0.518    0.612
## gini_index                               2.002e+01  1.026e+01   1.951    0.070
## perc_non_white                          -8.127e-01  1.862e+00  -0.436    0.669
##                                          
## (Intercept)                             .
## urbanizationlow                          
## median_household_income                  
## perc_population_with_high_school_degree  
## perc_non_citizen                         
## gini_index                              .
## perc_non_white                           
## ---
## Signif. codes:  0 '***' 0.001 '**' 0.01 '*' 0.05 '.' 0.1 ' ' 1
## 
## Residual standard error: 0.5112 on 15 degrees of freedom
## Multiple R-squared:  0.3469, Adjusted R-squared:  0.08563 
## F-statistic: 1.328 on 6 and 15 DF,  p-value: 0.3047
\end{verbatim}

\begin{Shaded}
\begin{Highlighting}[]
\NormalTok{unemployment_high_lm =}\StringTok{ }\KeywordTok{lm}\NormalTok{(}\KeywordTok{log}\NormalTok{(hate_crimes_per_100k_splc)}\OperatorTok{~}\NormalTok{.}\OperatorTok{-}\NormalTok{state, }\DataTypeTok{data =}\NormalTok{ unemployment_high)}
\KeywordTok{summary}\NormalTok{(unemployment_high_lm)}
\end{Highlighting}
\end{Shaded}

\begin{verbatim}
## 
## Call:
## lm(formula = log(hate_crimes_per_100k_splc) ~ . - state, data = unemployment_high)
## 
## Residuals:
##     Min      1Q  Median      3Q     Max 
## -1.3070 -0.3050 -0.0516  0.3564  1.0279 
## 
## Coefficients:
##                                           Estimate Std. Error t value Pr(>|t|)
## (Intercept)                             -2.722e+01  1.006e+01  -2.707   0.0156
## urbanizationlow                          2.858e-01  4.470e-01   0.639   0.5317
## median_household_income                 -3.394e-05  3.327e-05  -1.020   0.3229
## perc_population_with_high_school_degree  2.065e+01  9.995e+00   2.066   0.0554
## perc_non_citizen                         8.021e+00  9.071e+00   0.884   0.3896
## gini_index                               2.044e+01  9.414e+00   2.171   0.0453
## perc_non_white                          -4.979e-01  1.638e+00  -0.304   0.7651
##                                          
## (Intercept)                             *
## urbanizationlow                          
## median_household_income                  
## perc_population_with_high_school_degree .
## perc_non_citizen                         
## gini_index                              *
## perc_non_white                           
## ---
## Signif. codes:  0 '***' 0.001 '**' 0.01 '*' 0.05 '.' 0.1 ' ' 1
## 
## Residual standard error: 0.6922 on 16 degrees of freedom
## Multiple R-squared:  0.3624, Adjusted R-squared:  0.1233 
## F-statistic: 1.516 on 6 and 16 DF,  p-value: 0.2357
\end{verbatim}

Median household income coefficients have different signs when
unemployment is low vs high

\begin{Shaded}
\begin{Highlighting}[]
\NormalTok{urbanization_low =}\StringTok{ }\NormalTok{hate_nona_df }\OperatorTok\StringTok{ }
\StringTok{  }\KeywordTok{filter}\NormalTok{(urbanization }\OperatorTok{==}\StringTok{ "low"}\NormalTok{) }\OperatorTok\StringTok{ }
\StringTok{  }\KeywordTok{select}\NormalTok{(}\OperatorTok{-}\NormalTok{urbanization)}

\NormalTok{urbanization_high =}\StringTok{ }\NormalTok{hate_nona_df }\OperatorTok\StringTok{ }
\StringTok{  }\KeywordTok{filter}\NormalTok{(urbanization }\OperatorTok{==}\StringTok{ "high"}\NormalTok{) }\OperatorTok\StringTok{ }
\StringTok{  }\KeywordTok{select}\NormalTok{(}\OperatorTok{-}\NormalTok{urbanization)}

\CommentTok{# Stratified analysis for urbanization}

\NormalTok{urbanization_low_lm =}\StringTok{ }\KeywordTok{lm}\NormalTok{(}\KeywordTok{log}\NormalTok{(hate_crimes_per_100k_splc)}\OperatorTok{~}\NormalTok{.}\OperatorTok{-}\NormalTok{state, }\DataTypeTok{data =}\NormalTok{ urbanization_low)}
\KeywordTok{summary}\NormalTok{(urbanization_low_lm)}
\end{Highlighting}
\end{Shaded}

\begin{verbatim}
## 
## Call:
## lm(formula = log(hate_crimes_per_100k_splc) ~ . - state, data = urbanization_low)
## 
## Residuals:
##      Min       1Q   Median       3Q      Max 
## -0.94281 -0.32039 -0.02037  0.44346  0.90076 
## 
## Coefficients:
##                                           Estimate Std. Error t value Pr(>|t|)
## (Intercept)                             -1.748e+01  1.649e+01  -1.060    0.307
## unemploymentlow                          3.375e-03  4.615e-01   0.007    0.994
## median_household_income                 -1.582e-05  3.439e-05  -0.460    0.652
## perc_population_with_high_school_degree  1.150e+01  1.214e+01   0.947    0.360
## perc_non_citizen                        -1.703e+00  1.608e+01  -0.106    0.917
## gini_index                               1.560e+01  1.961e+01   0.796    0.439
## perc_non_white                          -7.016e-01  1.908e+00  -0.368    0.719
## 
## Residual standard error: 0.6062 on 14 degrees of freedom
## Multiple R-squared:  0.1546, Adjusted R-squared:  -0.2077 
## F-statistic: 0.4266 on 6 and 14 DF,  p-value: 0.8492
\end{verbatim}

\begin{Shaded}
\begin{Highlighting}[]
\NormalTok{urbanization_high_lm =}\StringTok{ }\KeywordTok{lm}\NormalTok{(}\KeywordTok{log}\NormalTok{(hate_crimes_per_100k_splc)}\OperatorTok{~}\NormalTok{.}\OperatorTok{-}\NormalTok{state, }\DataTypeTok{data =}\NormalTok{ urbanization_high)}
\KeywordTok{summary}\NormalTok{(urbanization_high_lm)}
\end{Highlighting}
\end{Shaded}

\begin{verbatim}
## 
## Call:
## lm(formula = log(hate_crimes_per_100k_splc) ~ . - state, data = urbanization_high)
## 
## Residuals:
##      Min       1Q   Median       3Q      Max 
## -1.39221 -0.29276  0.04506  0.28314  1.11647 
## 
## Coefficients:
##                                           Estimate Std. Error t value Pr(>|t|)
## (Intercept)                             -2.371e+01  7.403e+00  -3.203  0.00522
## unemploymentlow                          3.247e-01  2.889e-01   1.124  0.27665
## median_household_income                 -9.553e-06  2.410e-05  -0.396  0.69672
## perc_population_with_high_school_degree  1.801e+01  8.066e+00   2.233  0.03929
## perc_non_citizen                         1.995e+00  6.777e+00   0.294  0.77203
## gini_index                               1.441e+01  6.744e+00   2.137  0.04740
## perc_non_white                           1.197e+00  1.897e+00   0.631  0.53637
##                                           
## (Intercept)                             **
## unemploymentlow                           
## median_household_income                   
## perc_population_with_high_school_degree * 
## perc_non_citizen                          
## gini_index                              * 
## perc_non_white                            
## ---
## Signif. codes:  0 '***' 0.001 '**' 0.01 '*' 0.05 '.' 0.1 ' ' 1
## 
## Residual standard error: 0.6168 on 17 degrees of freedom
## Multiple R-squared:  0.4537, Adjusted R-squared:  0.2609 
## F-statistic: 2.353 on 6 and 17 DF,  p-value: 0.07723
\end{verbatim}

\begin{Shaded}
\begin{Highlighting}[]
\CommentTok{## Median household income coefficients have different magnitudes when urbanization is low vs high}
\end{Highlighting}
\end{Shaded}

Stratifying on unemployment we see that the coefficient for median
household income is positive when unemployment is low and negative when
unemployment is high. When we stratify on urbanization, we see that the
coefficient for median household income has a higher magnitude when
urbanization is low (although both coefficients are negative). These
stratified analyses indicate that interactions do exist.

Include these interaction terms in the full model to check for
significance.

\begin{Shaded}
\begin{Highlighting}[]
\NormalTok{interaction_log_lm =}\StringTok{ }\KeywordTok{lm}\NormalTok{(}
  \KeywordTok{log}\NormalTok{(hate_crimes_per_100k_splc)}
  \OperatorTok{~}\StringTok{ }\NormalTok{perc_population_with_high_school_degree}
  \OperatorTok{+}\StringTok{ }\NormalTok{gini_index}
  \OperatorTok{+}\StringTok{ }\NormalTok{median_household_income }
  \OperatorTok{+}\StringTok{ }\NormalTok{unemployment }\OperatorTok{+}\StringTok{ }\NormalTok{urbanization }
  \OperatorTok{+}\StringTok{ }\NormalTok{median_household_income}\OperatorTok{*}\NormalTok{unemployment }
  \OperatorTok{+}\StringTok{ }\NormalTok{median_household_income}\OperatorTok{*}\NormalTok{urbanization,}
  \DataTypeTok{data =}\NormalTok{ hate_nona_df)}

\KeywordTok{summary}\NormalTok{(interaction_log_lm)}
\end{Highlighting}
\end{Shaded}

\begin{verbatim}
## 
## Call:
## lm(formula = log(hate_crimes_per_100k_splc) ~ perc_population_with_high_school_degree + 
##     gini_index + median_household_income + unemployment + urbanization + 
##     median_household_income * unemployment + median_household_income * 
##     urbanization, data = hate_nona_df)
## 
## Residuals:
##      Min       1Q   Median       3Q      Max 
## -1.31163 -0.36613  0.00301  0.33587  1.09203 
## 
## Coefficients:
##                                           Estimate Std. Error t value Pr(>|t|)
## (Intercept)                             -1.964e+01  5.275e+00  -3.722 0.000655
## perc_population_with_high_school_degree  1.246e+01  4.738e+00   2.631 0.012346
## gini_index                               1.589e+01  5.906e+00   2.691 0.010641
## median_household_income                  2.032e-06  1.962e-05   0.104 0.918057
## unemploymentlow                         -3.161e-01  1.189e+00  -0.266 0.791826
## urbanizationlow                          1.333e+00  1.171e+00   1.138 0.262301
## median_household_income:unemploymentlow  9.229e-06  2.112e-05   0.437 0.664710
## median_household_income:urbanizationlow -2.716e-05  2.156e-05  -1.260 0.215666
##                                            
## (Intercept)                             ***
## perc_population_with_high_school_degree *  
## gini_index                              *  
## median_household_income                    
## unemploymentlow                            
## urbanizationlow                            
## median_household_income:unemploymentlow    
## median_household_income:urbanizationlow    
## ---
## Signif. codes:  0 '***' 0.001 '**' 0.01 '*' 0.05 '.' 0.1 ' ' 1
## 
## Residual standard error: 0.5737 on 37 degrees of freedom
## Multiple R-squared:  0.3435, Adjusted R-squared:  0.2193 
## F-statistic: 2.766 on 7 and 37 DF,  p-value: 0.02042
\end{verbatim}

Holding Gini index and percent of population with high school degree
constant, both interactions are not significant predictors of hate crime
incidents. They are not included in the final model.

\end{document}
